\indent \newline 

1. Om bedriften
AS ROCKWOOL er et heleid norsk datterselskap av ROCKWOOL International A/S. Konsernet er verdens største steinullprodusent med ca 10.600 ansatte, salgskontor i over 35 land, og 28 steinullfabrikker og 11 andre steinullrelaterte fabrikker i 17 land. Datterselskapet består i dag av 240 ansatte fordelt på administrasjon i Oslo, salgsapparatet og to fabrikker lokalisert henholdsvis i Moss og Trondheim. ROCKWOOL baserer sin virksomhet på utvinning av vulkansk stein for å produsere produkter, systemer og løsninger innenfor byggisolasjon.

Den første ROCKWOOL-fabrikken ble startet i 1937 i Danmark. Allerede året etter kom den første norske fabrikken i Larvik, og noen år senere utvidet de med fabrikk i Moss og Trondheim. I 1965 fusjonerte Elkem og ROCKWOOL, hvor Elkem i 1993 ble kjøpt ut av ROCKWOOL International A/S. ROCKWOOL har siden oppstarten produsert steinull, hvor selve produksjonsprosessen har vært relativt lik. Derimot ble det i 2002 investert en halv milliard kroner i produksjonsutstyr som ved hjelp av robotteknologi ga en helautomatisert produksjonsprosess. Dette har ført til mer enn en fordobling av produksjonen, og gitt helsegevinster ved at arbeidsoppgaver tilknyttet smelteprosessen har blitt erstattet av maskiner og datateknologi.

AS ROCKWOOL har i lengre tid gjort det bra, og kan vise til sterke nøkkeltall i form av god likviditet og meget god lønnsomhet og soliditet. Bedriften har i dag en markedsandel på ca 26% og omsatte for over 800 millioner kroner i 2017. Deres største konkurrent er Glava med en markedsandel på 40% og en omsetning på over 1,3 milliarder kroner i samme år. Bransjen er preget av høye faste kostnader og store investeringer. Konkurransesituasjon er tøff, med et stadig økende fokus på miljøvennlige produkter og krav fra myndighetene. “I anbudsprosessen har AS ROCKWOOL uttalt at valget faller på leverandører som har lavest CO2-utslipp, da det gir en gevinst i ettertid for kundene.“

Bedriftens visjon er å være ledende leverandør av isolasjon, der positivt bidrag til et bedre miljø og brannsikring skal være førende. Dette er basert på FN sine mål?

Utfordringene ROCKWOOL står overfor er å finne nye måter å produsere på som er mer miljøvennlige for å kunne være konkurransedyktige også i fremtiden.

2. Metode
Jeg bestemte meg tidlig for å velge og skrive om AS ROCKWOOL, da jeg kjenner sjefen for fabrikken i Moss (Erik Ølstad). Dette ga meg muligheten til å holde en jevn dialog med en ansatt i bedriften som har meget god innsikt i hvordan bedriften drives og markedet den opererer i. Utover i prosessen fikk jeg anledning til å intervjue Lasse Scott (økonomiansvarlig) og Hans Joachim Motzfeldt (senior advisor). Ved å benytte en semistrukturert intervjuform, skapte dette en åpen samtale og anledning for intervjuobjektene til å reflektere rundt temaet og oppgaven. Økonomiske tall relatert til bedriften og dens konkurrenter er primært hentet fra proff.no. Annen informasjon er blitt hentet fra bedriftens hjemmeside og diverse artikler på nettet. Jeg har gjennom arbeidsprosessen aktivt benyttet analyser og metoder fra pensum.








3. Intern Analyse
3.1 Verdikonfigurasjons-analyse

Verdikonfigurasjonsanalysen identifiserer sammensetningen av bedriftens aktiviteter og tilhørende verdi- og kostnadsdrivere (s. 32). AS ROCKWOOL driver med industriell produksjon som er typisk for en verdikjede.

Primæraktiviteter
Primæraktivitene er et sekvensielt sett av aktiviteter som direkte skaper verdi for kunden (s.132). 
  
Som vist i figuren over er den første aktiviteten inngående logistikk. Råvarer som vulkansk stein, koks og slagg (avfallsprodukter fra aluminiumsproduksjon) lagres, før det fraktes videre til produksjon. Her omformes innsatsfaktorene fra råvarer til steinull ved å bli utsatt for enormt høye temperaturer i en smelteovn. De ferdige produktene fraktes videre til lagring eller transporteres direkte til kunder avhengig av om det er produsert på bestilling. Markedsføring og salg fokuserer på byggevarekjedene og entrepenørene. Det markedsføres derfor ikke direkte mot privatpersoner. Kundeservice består i hovedsak av teknisk service.
ROCKWOOL har utfordringer med høye transportkostnader i forhold til andre konkurrenter grunnet hvor godt produktene lar seg komprimere. Produktene transporteres med lastebil ved fastprisavtale, og dårligere komprimering av produktene gir lavere volum per transport. 

Støtteaktiviteter
Støtteaktivitetene skaper indirekte verdi for kunden gjennom sin effekt på primæraktivitetene (s.132). 

Innkjøp
AS ROCKWOOL sin størrelse og del av konsern gir gjennom sentral koordinering og store innkjøpskvantum, fordelaktige innkjøpsavtaler.

Teknologi og utvikling
Både i AS ROCKWOOL og konsernet har de i mange år lagt mye ressurser i teknologiutvikling. De har tidligere utviklet sitt eget bindemiddel som gir konkurransefortrinn i form av et produkt som har et større anvendelsesområde enn å bare isolere. I dag jobbes det med å utvikle produksjonsutstyr som kan bidra til å produsere et mer miljøvennlig produkt.

Personalforvaltning
De ansatte trives godt i jobben og føler tilhørighet til arbeidsoppgavene. Det er innarbeidet gode rutiner som sørger for at det er to personer med samme kompetanse til å utføre hver arbeidsoppgave. Dette sørger for en kontinuerlig flyt selv ved sykefravær og eventuelle oppsigelser. 

Infrastruktur
Vedlikehold og drift av produksjonen vurderes regelmessig etter KPI nøkkeltall som iverksetter . De større utstyr investeringer bærer preg av høye driftskostnader og overordnet startegi går på å nå måltall, markedsandeler, omsetning, vekst, bærekraft, digital platform, salgssystemer, produktutvikling, cost-saving


ROCKWOOL har i dag to fabrikker lokalisert henholdsvis i Moss og Trondheim, og en administrasjon og salgsapparat i Oslo. En klar og overordnet visjon blir kommunisert godt gjennom alle avdelinger som sørger for en mest mulig effektiv drift. Må skrive noe annet også.
  
Oppsummering?
AS ROCWOOL har i flere år jobbet med å effektivisere verdiekjeden. De har implementert en lean-metode som de kaller Ropex. (https://www.dagsavisen.no/moss/lokalt/viktig-for-industribyen-moss-1.316778).
Bedriften kjennetegnes derfor av god kommunikasjon på tvers av avdelinger og kommandolinjer med fokus på å produsere mest mulig effektivt. På bakgrunn av dette, ser man lite rom for forbedring i verdikjeden. ROCKWOOL bør imidlertid fortsette å investere i teknologiutvikling for å kunne forbedre kompaktheten til produktene og dermed øke volumet per transport.

3.2 Verdikjedens drivere
Dette er strukturelle faktorer som påvirker verdiskapning for kunden og enhetskostnadene forbundet med å utføre aktivitetene.(s.32) De viktigste driverne for AS ROCKWOOL er stordriftsfordeler og kapasitetsutnyttelse. Bedriften kan lagre produktene i opp til ett år før de må gjennom en ny kvalitetskontroll. Lang holdbarhet på produktene gir mulighet til å produsere i stor skala og dermed senke enhetskostnadene.

Stordriftsfordelene forsterkes også av å være en del av ROCKWOOL-konsernet. Bedriften drar nytte av operasjonelle synergier gjennom gunstige prisavtaler hos leverandører ved å handle i stort kvantum.

3.3 VRIO-analyse

Ressurs
Verdifull
Sjelden
Vanskelig å kopiere
Effektivt organisert
Avkastning
Finansiell kapital
Ja
nei
nei
Ja
Over gjennomsnittet
Kompetanse
ja
ja
ja
ja
Over gjennomsnittet
Teknologi
ja
nei
nei
nei
Litt under gjennomsnittet
Produktegenskaper
ja
nei
ja
ja
Over gjennomsnittet

3.3.1 Finansiell kapital
AS ROCKWOOL har hatt en lønnsom drift i flere år og hadde i 2017 et årsresultat på 64 millioner kr og en egenkapital på 382 millioner kr.(ref proff.no) I tillegg drar de nytte av finansielle synergier ved å være en del av et konsern. De står dermed godt rustet for potensielle investeringer i tiden fremover.

3.3.2 Kompetanse
Kompetansen anses som meget god. Turnover-raten er på bare 2% og mange av de ansatte har en fartstid på 15-35 år i bedriften. For å fortsette og utvikle kompetansen blant de ansatte, har de innarbeidet interne opplærings- og utdanningssystemer. Bedriften bør imidlertid være oppmerksom på at mye av kompetansen kan forsvinne i årene som kommer på grunn av en relativt høy gjennomsnittsalder hos de ansatte.

3.3.3 Teknologi
Det investeres mye ressurser i teknologi, både i AS ROCKWOOL og konsernet. Ved å utvikle egne smelteovner har dette tidligere gitt konkurransefortrinn i markedet, men med dagens smelteteknologi henger de etter de største konkurrentene når det gjelder å produsere miljøvennlig. Å redusere CO2-utslipp i forbindelse med produksjonsprosessen anses som kritisk for å kunne være konkurransedyktige i fremtiden.

3.3.4 Produktegenskaper
Steinull innehar flere egenskaper og bruksområder enn de fleste andre isolasjonsproduktene som tilbys i markedet. Produktet isolerer, er vannavstøtende, har lyddempende egenskaper og er en god kilde til brannsikring. Det er kun ROCKWOOL og Paroc som scorer høyt på alle disse produktatributtene, mens andre aktører kun leverer på to av punktene.  

3.4 Oppsummering (sensorveiledning - gode oppgaver oppsummerer hvilke ressurser som gir konk.fortrinn - hva er viktig fremover?)
Analysen viser at ROCKWOOL, sammen med Paroc, har konkurransefortrinn i produktegenskapene. Dette anses som langvarig da egenskapene kommer naturlig fra råvaren, og det vil også være kostbart for andre aktører å bytte til steinullproduksjon i form av store investeringer og mangel på erfaring. Det vil være viktig fremover å utvikle og/eller investere i ny miljøvennlig smelteteknologi . Dette anses som bedriftens største utfordring i dag.

4. Ekstern analyse
Denne seksjonen tar for seg Porters fem krefter og PESTEL- analyse, som gir innsikt i byggisolasjonsbransjen og bedriftens makroomgivelser. Analysene kartlegger eventuelle trusler og muligheter bedriften står overfor i tiden fremover.

Konkurransearena
AS ROCKWOOL sin konkurransearena er byggisolasjon og er avgrenset til det norske og svenske markedet som bedriften leverer til.  Konkurrenter er henholdsvis de samme i både Norge og Sverige, med Glava, Paroc, Knauf og Sundolitt som de største. Glava produserer glassullisolasjon med størst markedsandel på 40% (ref tabell og økonomiske tall). Paroc er AS ROCKWOOL sin mest nærliggende konkurrent da de produserer steinullisolasjon. De siste årene har det kommet nye alternative isolasjonsprodukter som halmtekstil, papir og trefiber.




Råvare
Isolasjon
Vannavstøtende
Lydisolasjon
Brannsikring
ROCKWOOL
Steinull
+
+
+
+
Paroc
Steinull
+
+
+
+
Glava
Glassull
+
-
-
+
Knauf
Glassull
+
-
-
+
Sundolitt
Plastull
+
+
-
-


Bedriftens strategiske gruppe identifiseres ved deres konkurranse og tilnærming til kunder (s.88). Samtlige aktører i markedet konkurrerer om de samme kundene, som i hovedsak er større entreprenører, byggevarekjeder og ferdige hytte/hus-produsenter. Derimot gjør produktegenskapene (ref produktegenskaper) at det er naturlig å dele aktørene inn i to ulike strategigrupper. Paroc er den bedriften som likner mest, og plasseres derfor i samme gruppe som ROCKWOOL, mens Glava, Knauf og Sundolitt plasseres i en annen. 
 

4.1 Porters fem krefter
(Tabell)

4.1.1 Trussel fra inntrengere
Inntrengere er mulige nye konkurrenter som ønsker å etablere seg i byggisolasjonsbransjen.(s. 94) I en etableringsprosess for nye aktører tar man for seg mobilitetsbarriererene som påvirker inngang og utgang fra bransjen.(s 90) Kapitalbehovet er stort, da det kreves store investeringer i spesialisert produksjonsutstyr. AS ROCKWOOL og de andre største aktørene  har gjort disse investeringene over tid, noe som gir dem konkurransefortrinn overfor inntrengere. Inntrengere må også ta hensyn til de store avviklingsbarrierene og høye faste kostnader. Stordriftsfordelene blant de største aktørene er også med på å redusere antall konkurrenter. Inngangsbarriererene er dermed høye og trussel fra inntrengere regnes som lav.
 
4.1.2 Trussel fra substitutter
Substituttene til AS ROCKWOOL er andre byggisolasjonsprodukter som dekker de samme behovene til kunden. Halm, papir og trefiber er alternative produkter, men har ikke nevneverdig markedsandel. Det vil allikevel være viktig å følge med på teknologiutviklingen som kan gjøre disse produktene mer konkurransedyktige i fremtiden.


4.1.3 Kundenes forhandlingsmakt
Bransjen består i hovedsak av tre ulike kunder; byggevarekjeder, hytte/hus-produsenter og entrepenører. Byggevarekjedene fokuserer mest på pris, men langsiktige avtaler kan også forhandles frem ved bruk av bonuser og krav til å holde seminarer for byggevarekjedenes kunder. Entrepenørene fokuserer ikke bare på pris, men også hvor miljøvennlige produktene er. De står overfor strenge krav i forbindelse med totalt CO2-utslipp i byggeprosessen. De vil derfor ofte favorisere produkter med lavest CO2-utslipp for å spare kostnader.
 
Byggevarekjedene og entreprenørene er konsentrerte og har dermed høy forhandlingsmakt. Det er også relativt lite differensierte produkter som gir lave byttekostnader. Ferdige hus- og hytteprodusenter er ikke like store og har derfor mindre forhandlingsmakt. 

4.1.4 Leverandørenes forhandlingsmakt
Det finnes mange leverandører av råvarer i markedet, og flere av isolasjonsprodusentene handler i stort kvantum. Dette gir høye byttekostnader og dermed lav forhandlingsmakt.

4.1.5 Intern rivalisering
Isolasjonsbransjen er et modent marked. De siste årene har markedsveksten ligget på ca. 2%, og prognosene tyder på at dette vil fortsette i årene som kommer. Markedet består i hovedsak av noen få store aktører som omsetter for flere hundre millioner kroner. Produktene som tilbys er lite differensierte, bortsett fra Paroc og ROCKWOOL som tilbyr produkter med flere egenskaper. Analysen av konkurransearenaen viser at skal ROCKWOOL kunne ta større markedsandeler, må dette gjøres på bekostning av konkurrentene.

4.2 PESTEL-analyse


Politiske
Teknologiske
Økonomiske
Miljømessige
Sosiokulturelle
Legale








Urbanisering




4.2.1 Politiske faktorer - signaler fra myndighetene om standarder vedrørende miljøet, få støtte av enova, skifte til miljøvennlige bygninger

4.2.2 Teknologiske faktorer - krav til raskerer bygging, flyttes fra snekkere til ferdighusprodusenter 

4.2.3 Økonomiske faktorer - byggetid

4.2.4 Miljømessige faktorer

4.2.5 Sosiokulturelle faktorer 
På bakgrunn av innvandring og en trend blant unge, velger flere å flytte inn til byene. https://www.dagsavisen.no/innenriks/vi-flytter-til-byene-og-vraker-distriktene-1.1099412
Utviklingen går derfor mot mer bebyggelse av leilighetskomplekser og færre eneboliger. På sikt vil 

- folk flytter inn til byene, går fra eneboliger til leilighetskomplekser - flytter salget fra byggevarekjedene til entrepenørene

4.2.6 Legale faktorer - lovpålagte byggekrav, byggeforskrifter

5. SWOT-analyse

6. Strategiske handlingsalternativ

7. Anbefalt strategi + implementering







Strategiske valg:
1. Investere i ny smelteovn - bli enda mer miljøvennlig
2. fusjonere/oppkjøp
3. Outsource produksjon til danmark
4. 


Kildeliste
Konkurrenter:
Glava er også en del av et konsern
Sundolitt - samlebetegnelse på isolasjon basert på plast - 11, trefibre, papir, halm, 
Glava, rockwool, parock, knauf - største aktører

Glass og stein størst - mineralull

Halvparten av produksjon går til sverige - samme konkurrenter på det markedet

Kjøper stein lenger nord enn trondheim - gudvangen og hurum - juve steinbrudd, 

Koks fra spania - Nalon
Avfallsprodukter fra aluminiumsproduksjon - kilder for å få høyere aluminiumsinnhold i produktet (riktig sammensetning) - leverandør merocks slagg (sverige), Hydro



Sluttransport er dyr

Kundenes forhandlingsmakt er stor -

Steinull har konkurransefortrinn overfor glassull i forhold til brannsikring. (tar ca en halvtime lenger for en brann å spre seg med steinull)
Steinull - brukes til å isolere rundt stålbjelkene i bygningskonstruksjoner da stålet vil smelte før steinullen. 
Steinull tåler ca 1000 grader før det smelter (tar også mye lenger tid)
Andre mineralull (glass) 600-700
Stål litt over 500
Plastikk mellom 100 og 400

I gjennomsnitt er produsert steinull laget av ⅓ upcycled waste
38 000 ganger mer stein blir produsert i forhold til det som går til å produsere rockwool stein ull
I 2017 hadde de en recycled content opp til 50 %, globalt gjennomsnitt på 31%

EPD

Egen innkjøpsavdeling, support-avdeling, finansavdeling, teknologisk utvikling på produkt og maskin/utstyr, egen gruppe basisdata til salg/produksjon, IT-avdeling, HR, 

Produktets andvendelse: isolasjon, bærende isolasjon (flate tak), løse fibre som kan være med på å forsterke

Kompetanse meget god, folk som har vært der i 15-20 år, lav turnover (2%), to mann til hver jobb, dvs to personer har kompetanse til å gjøre en jobb, utdanner ansatte internt, opptatt av å utvikle ansatte

Produktegenskaper ROCKWOOL - isolerer, støydemping (lydisolasjon), brannsikring og vannavvisende, mens konkurrenter stort sett bare kan levere to av egenskapene i ett produkt. 

Svakhet: Klarer ikke å komprimere materialet like godt som glassull, dette gjør at transportkostnadene blir høyere da det er fast pris per lastebil og får ikke like mye volum inn per last 

Styrke: dekker alle bruksområder

Trussler: Miljøkrav, CO2-utslipp

Muligheter: 


