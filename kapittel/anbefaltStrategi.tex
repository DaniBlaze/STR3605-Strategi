På bakgrunn av analysene jeg har gjennomført, mener jeg at en kombinasjon av handlingsalternativene “Fortsette med organisk vekst” og “Go Green” vil kunne hjelpe ROCKWOOL med å nå sin visjon.

\indent \newline
Økonomiske nøkkeltall viser til en bedrift som har prestert godt over tid, noe som er et resultat av flere år med fokus på å effektivisere verdikjeden. Implementeringen av Ropex har vært en viktig faktor i dette arbeidet, og det vil derfor være naturlig å fortsette med fokus på organisk vekst. På bakgrunn av markedsanalysen vil jeg imidlertid anbefale ROCKWOOL til å tillegge miljøet større vekt i deres nåværende strategi, hvor bedriften bør jobbe mot et mål om å bli en av markedets mest miljøvennlige aktører. Ved å investere i nytt produksjonsutstyr vil ROCKWOOL kunne utvide markedsandelen sin ved å bli en foretrukket leverandør blant entreprenørene, og på sikt utfordre Glava som markedsleder.

\indent \newline
For å evaluere strategien vil jeg benytte SAF-kriteriene; egnethet, aksept og gjennomførbarhet \cite[s.~236-241]{FundamentalsOfStrategy}. Kriteriene sørger for å kvalitetssikre strategien. Den anbefalte løsningen samsvarer med markedsutviklingen og de største truslene bedriften står overfor i dag. ROCKWOOL sine interessenter vil med stor sannsynlighet være åpne for endringen. De ansatte vil måtte forholde seg til nye KPIer og måltall, men arbeidsoppgavene vil være de samme. Investeringen medfører risiko i form av en oppstartsperiode hvor tilpasning og utforskning blir viktig for å integrere den nye teknologien. Teknologien er ikke tatt i bruk i samme skala tidligere. Etter samtaler med Erik Ølstad vil investeringen kunne koste opp mot en halv milliard kroner og være bedriftsøkonomisk ulønnsom de første årene i form av høyere avskrivninger og nevnte oppstartsfase. Det vil med stor sannsynlighet ta tid før optimal tilpasning i produksjonsprosessen nås. Prosjektet anses som gjennomførbart med støtte fra Enova. Hvis det ikke oppnås støtte fra Enova, presiseres det at det kreves finansiell støtte fra ROCKWOOL international AS.