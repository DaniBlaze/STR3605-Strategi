3. Intern Analyse
3.1 Verdikonfigurasjons-analyse

Verdikonfigurasjonsanalysen identifiserer sammensetningen av bedriftens aktiviteter og tilhørende verdi- og kostnadsdrivere (s. 32). ROCKWOOL driver med industriell produksjon som er typisk for en verdikjede.

Primæraktiviteter
Primæraktivitene er et sekvensielt sett av aktiviteter som direkte skaper verdi for kunden (s.132). 
  
Som vist i figuren over er den første aktiviteten inngående logistikk. Råvarer som vulkansk stein, koks og slagg (avfallsprodukter fra aluminiumsproduksjon) lagres, før det fraktes videre til produksjon. Her omformes innsatsfaktorene fra råvarer til steinull ved å bli utsatt for enormt høye temperaturer i en smelteovn. De ferdige produktene fraktes videre til lagring eller transporteres direkte til kunder avhengig av om det er produsert på bestilling. Markedsføring og salg fokuserer på byggevarekjedene og entreprenørene. Det markedsføres derfor ikke direkte mot privatpersoner. Kundeservice består i hovedsak av teknisk service.
ROCKWOOL har utfordringer med høye transportkostnader i forhold til andre konkurrenter grunnet hvor godt produktene lar seg komprimere. Produktene transporteres med lastebil ved fastprisavtale, og dårligere komprimering av produktene gir lavere volum per transport. 

Støtteaktiviteter
Støtteaktivitetene skaper indirekte verdi for kunden gjennom sin effekt på primæraktivitetene (s.132). 

Innkjøp
AS ROCKWOOL sin størrelse og del av konsern gir gjennom sentral koordinering og store innkjøpskvantum, fordelaktige innkjøpsavtaler.

Teknologi og utvikling
Både i AS ROCKWOOL og konsernet har de i mange år lagt mye ressurser i teknologiutvikling. De har tidligere utviklet sitt eget bindemiddel som gir konkurransefortrinn i form av et produkt som har et større anvendelsesområde enn å bare isolere. I dag jobbes det med å utvikle produksjonsutstyr som kan bidra til å produsere et mer miljøvennlig produkt.

Personalforvaltning
De ansatte trives godt i jobben og føler tilhørighet til arbeidsoppgavene. Det er innarbeidet gode rutiner som sørger for at det er to personer med samme kompetanse til å utføre hver arbeidsoppgave. Dette sørger for en kontinuerlig flyt selv ved sykefravær og eventuelle oppsigelser. 

Infrastruktur
Infrastruktur: handler om støttesystemer, og de funksjonene som gjør at daglig drift opprettholdes. Regnskap, juridiske forhold, administrasjon og daglig ledelse er eksempler på nødvendig infrastruktur som organisasjoner kan bruke til sin fordel.

Vedlikehold og drift av produksjonen vurderes regelmessig etter KPIer.nøkkeltall som iverksetter . De større utstyr investeringer bærer preg av høye driftskostnader og overordnet strategi går på å nå måltall, markedsandeler, omsetning, vekst, bærekraft, digital platform, salgssystemer, produktutvikling, cost-saving
  
Oppsummering
AS ROCWOOL har i flere år jobbet med å effektivisere verdiekjeden. De har implementert en lean-metode som de kaller Ropex. (https://www.dagsavisen.no/moss/lokalt/viktig-for-industribyen-moss-1.316778).
Bedriften kjennetegnes derfor av god kommunikasjon på tvers av avdelinger og kommandolinjer med fokus på å produsere mest mulig effektivt. På bakgrunn av dette, ser man lite rom for forbedring i verdikjeden. ROCKWOOL bør imidlertid fortsette å investere i teknologiutvikling for å kunne forbedre kompaktheten til produktene og dermed øke volumet per transport.

3.2 Verdikjedens drivere
Dette er strukturelle faktorer som påvirker verdiskapning for kunden og enhetskostnadene forbundet med å utføre aktivitetene.(s.32) De viktigste driverne for ROCKWOOL er stordriftsfordeler og kapasitetsutnyttelse. Bedriften kan lagre produktene i opp til ett år før de må gjennom en ny kvalitetskontroll. Lang holdbarhet på produktene gir mulighet til å produsere i stor skala og dermed senke enhetskostnadene.

Stordriftsfordelene forsterkes også av å være en del av ROCKWOOL-konsernet. Bedriften drar nytte av operasjonelle synergier gjennom gunstige prisavtaler hos leverandører ved å handle i stort kvantum.

3.3 VRIO-analyse

Ressurs
Verdifull
Sjelden
Vanskelig å kopiere
Effektivt organisert
Avkastning
Finansiell kapital
Ja
nei
nei
Ja
Over gjennomsnittet
Kompetanse
ja
ja
ja
ja
Over gjennomsnittet
Teknologi
ja
nei
nei
nei
Litt under gjennomsnittet
Produktegenskaper
ja
nei
ja
ja
Over gjennomsnittet

3.3.1 Finansiell kapital
AS ROCKWOOL har hatt en lønnsom drift i flere år og hadde i 2017 et årsresultat på 64 millioner kr og en egenkapital på 382 millioner kr.(ref proff.no) I tillegg drar de nytte av finansielle synergier ved å være en del av et konsern. De står dermed godt rustet for potensielle investeringer i tiden fremover.

3.3.2 Kompetanse
Kompetansen anses som meget god. Turnover-raten er på bare 2% og mange av de ansatte har en fartstid på 15-35 år i bedriften. For å fortsette og utvikle kompetansen blant de ansatte, har de innarbeidet interne opplærings- og utdanningssystemer. Bedriften bør imidlertid være oppmerksom på at mye av kompetansen kan forsvinne i årene som kommer på grunn av en relativt høy gjennomsnittsalder hos de ansatte.

3.3.3 Teknologi
Det investeres mye ressurser i teknologi, både i AS ROCKWOOL og konsernet. Ved å utvikle egne smelteovner har dette tidligere gitt konkurransefortrinn i markedet, men med dagens smelteteknologi henger de etter de største konkurrentene når det gjelder å produsere miljøvennlig. Å redusere CO2-utslipp i forbindelse med produksjonsprosessen anses som kritisk for å kunne være konkurransedyktige i fremtiden.

3.3.4 Produktegenskaper
Steinull innehar flere egenskaper og bruksområder enn de fleste andre isolasjonsproduktene som tilbys i markedet. Produktet isolerer, er vannavstøtende, har lyddempende egenskaper og er en god kilde til brannsikring. Det er kun ROCKWOOL og Paroc som scorer høyt på alle disse produktatributtene, mens andre aktører kun leverer på to av punktene.  

3.4 Oppsummering 
Analysen viser at ROCKWOOL, sammen med Paroc, har konkurransefortrinn i produktegenskapene. Dette anses som langvarig da egenskapene kommer naturlig fra råvaren, og det vil også være kostbart for andre aktører å bytte til steinullproduksjon i form av store investeringer og mangel på erfaring. Det vil være viktig fremover å utvikle og investere i ny miljøvennlig smelteteknologi . Dette anses som bedriftens største utfordring i dag.

4. Ekstern analyse
Denne seksjonen tar for seg Porters fem krefter og PESTEL- analyse, som gir innsikt i byggisolasjonsbransjen og bedriftens makroomgivelser. Analysene kartlegger eventuelle trusler og muligheter bedriften står overfor i tiden fremover.

Konkurransearena


Konkurrent
Driftsresultat 
Markedsandel




Glava








Knauf








Paroc








Sundolitt
119 000 000







AS ROCKWOOL sin konkurransearena er byggisolasjon og er avgrenset til det norske og svenske markedet som bedriften leverer til.  Konkurrenter er henholdsvis de samme i både Norge og Sverige, med Glava, Paroc, Knauf og Sundolitt som de største. Glava produserer glassullisolasjon med størst markedsandel på 40% (ref tabell og økonomiske tall). Paroc er AS ROCKWOOL sin mest nærliggende konkurrent da de produserer steinullisolasjon. De siste årene har det kommet nye alternative isolasjonsprodukter som halmtekstil, papir og trefiber.



Råvare
Isolasjon
Vannavstøtende
Lydisolasjon
Brannsikring
ROCKWOOL
Steinull
+
+
+
+
Paroc
Steinull
+
+
+
+
Glava
Glassull
+
-
-
+
Knauf
Glassull
+
-
-
+
Sundolitt
Plastull
+
+
-
-


Bedriftens strategiske gruppe identifiseres ved deres konkurranse og tilnærming til kunder (s.88). Samtlige aktører i markedet konkurrerer om de samme kundene, som i hovedsak er større entreprenører, byggevarekjeder og ferdige hytte/hus-produsenter. Derimot gjør produktegenskapene (ref produktegenskaper) at det er naturlig å dele aktørene inn i to ulike strategigrupper. Paroc er den bedriften som likner mest, og plasseres derfor i samme gruppe som ROCKWOOL, mens Glava, Knauf og Sundolitt plasseres i en annen. 
 
4.1 Porters fem krefter
(Tabell)

4.1.1 Trussel fra inntrengere
Inntrengere er mulige nye konkurrenter som ønsker å etablere seg i byggisolasjonsbransjen.(s. 94) I en etableringsprosess for nye aktører tar man for seg mobilitetsbarriererene som påvirker inngang og utgang fra bransjen.(s 90) Kapitalbehovet er stort, da det kreves store investeringer i spesialisert produksjonsutstyr. AS ROCKWOOL og de andre største aktørene  har gjort disse investeringene over tid, noe som gir dem konkurransefortrinn overfor inntrengere. Inntrengere må også ta hensyn til de store avviklingsbarrierene og høye faste kostnader. Stordriftsfordelene blant de største aktørene er også med på å redusere antall konkurrenter. Inngangsbarriererene er dermed høye og trussel fra inntrengere regnes som lav.
 
4.1.2 Trussel fra substitutter
Substituttene til AS ROCKWOOL er andre byggisolasjonsprodukter som dekker de samme behovene til kunden. Halm, papir og trefiber er alternative produkter, men har ikke nevneverdig markedsandel. Det vil allikevel være viktig å følge med på teknologiutviklingen som kan gjøre disse produktene mer konkurransedyktige i fremtiden.

4.1.3 Kundenes forhandlingsmakt
Bransjen består i hovedsak av to ulike kunder; byggevarekjeder og entreprenører. Byggevarekjedene fokuserer mest på pris, men langsiktige avtaler kan også forhandles frem ved bruk av bonuser og krav til å holde seminarer for byggevarekjedenes kunder. Entreprenørene fokuserer ikke bare på pris, men også hvor miljøvennlige produktene er. De står overfor strenge krav i forbindelse med totalt CO2-utslipp i byggeprosessen. De vil derfor ofte favorisere produkter med lavest CO2-utslipp for å spare kostnader.
 Byggevarekjedene og entreprenørene er konsentrerte og har dermed høy forhandlingsmakt. Det er også relativt lite differensierte produkter som gir lave byttekostnader. 

4.1.4 Leverandørenes forhandlingsmakt
Det finnes mange leverandører av råvarer i markedet, og flere av isolasjonsprodusentene handler i stort kvantum. Dette gir høye byttekostnader og dermed lav forhandlingsmakt.

4.1.5 Intern rivalisering
Isolasjonsbransjen er et modent marked. De siste årene har markedsveksten ligget på ca. 2%, og prognosene tyder på at dette vil fortsette i årene som kommer. Markedet består i hovedsak av noen få store aktører som omsetter for flere hundre millioner kroner. Produktene som tilbys er lite differensierte, bortsett fra Paroc og ROCKWOOL som tilbyr produkter med flere egenskaper. Analysen av konkurransearenaen viser at skal ROCKWOOL kunne ta større markedsandeler, må dette gjøres på bekostning av konkurrentene.

4.2 PESTEL-analyse
Denne delen identifiserer faktorer i ROCKWOOL sine makroomgivelser, som vil kunne påvirke bedriften i tiden fremover. Nedenfor presenteres de viktigste funnene. 

Politiske
Teknologiske
Økonomiske
Miljømessige
Sosiokulturelle
Legale
Enova
Elektrisk transport
-Oppgangskonjunktur
-Rentehevinger
Grønt skifte
BREEAM
Urbanisering
CO2-avgift

4.2.1 Politiske faktorer 
Enova forvalter midlene i energifondet. Formålet deres er å drive bransjene i Norge mot et lavutslippssamfunn. De tilbyr derfor økonomisk støtte til bedrifter som ønsker å velge mer energi- og klimavennlige løsninger. ROCKWOOL bør se på muligheten for støtte ved en eventuell investering i nytt produksjonsutstyr.
https://www.enova.no/om-enova/

4.2.2 Teknologiske faktorer
De siste årene har smelteteknologien utviklet seg fra fra kupolovner til el-ovner med et formål om å redusere utslipp.Bruk av elektriske lastebiler vil de neste årene bli viktig for transportselskaper. Denne utviklingen bør ROCKWOOL følge nøye med på og kan gi muligheter til å skape en mer  miljøvennlig profil.  

4.2.3 Økonomiske faktorer 
Norsk økonomi har vært i moderat oppgangskonjunktur det siste halvannet året og forventes å gå inn i en høykonjunktur fremover. BNP økte i 2017 med 2% og har fortsatt med en moderat vekst så langt i 2018. Arbeidsledigheten har falt og indikasjoner tyder på at den vil fortsette å synke. Dette er faktorer som reflekterer en forbedret materiell velstand, noe som er positivt for ROCKWOOL. Derimot vil årets renteheving og de planlagte rentehevingene i 2019 kunne påvirke folk sine muligheter for låneopptak og dermed virke negativt inn på nybygg-markedet. 
https://www.ssb.no/nasjonalregnskap-og-konjunkturer/artikler-og-publikasjoner/hoykonjunktur-i-sikte

4.2.4 Miljømessige faktorer
Samfunnet er inne i en utvikling hvor miljøet får en viktigere rolle, og hvor flere entreprenører velger å BREEAM-sertifisere prosjektene sine. BREAM er et miljøsertifiseringsverktøy for bygninger som påvirker ROCKWOOL negativt med dagens smelteteknologi. Bedriften kan stå i fare for å miste en betydelig markedsandel hvis de ikke tar hensyn til utviklingen. https://ngbc.no/breeam-nor/

4.2.5 Sosiokulturelle faktorer 
På bakgrunn av innvandring og en trend blant unge, velger flere å flytte inn til byene. Utviklingen går derfor mot mer bebyggelse av leilighetskomplekser og færre eneboliger. På sikt bør bedriftens kundefokus rette seg mer mot entreprenørene.
https://www.dagsavisen.no/innenriks/vi-flytter-til-byene-og-vraker-distriktene-1.1099412

4.2.6 Legale faktorer 
CO2-avgiften er en sentral kostnad for industriproduksjon. Med et økende miljøfokus og press i samfunnet, vil regjeringen kunne øke avgiftssatsen i nær fremtid. En eventuell økning vil treffe ROCKWOOL hardere enn flere av konkurrentene. 

https://www.regjeringen.no/no/tema/okonomi-og-budsjett/skatter-og-avgifter/veibruksavgift-pa-drivstoff/co2-avgiften/id2603484/

https://www.dagbladet.no/kultur/dagbladet-mener-co2-avgiften-ma-okes/70305954

5. Oppsummering med SWOT
Styrker
-Stordriftsfordeler
-Kompetanse (erfaring)
-Produktegenskaper (anvendelsesområder)
-Godt og innarbeidet opplærings- og utdanningssystem


Svakheter
-Produksjonsutstyr (Høyt CO2-utslipp)
- Høy gjennomsnittsalder hos ansatte
- Høye transportkostnader grunnet dårlig komprimering av produktene
Muligheter
-El-ovn
-Støtte fra Enova
-Elektriske lastebiler
-Urbanisering
-Oppgangskonjunktur (økt materiell velstand)
-Fusjon/oppkjøp (Paroc)


Trusler
-BREEAM
- Kundenes forhandlingsmakt
-Renteheving
- Økt CO2-avgift
- Modent marked





SWOT-analysen er en oppsummering av de interne og eksterne analysene. Formålet med analysen er å identifisere hvilke muligheter som ligger til grunn for fremtidig vekst, og trusler ROCKWOOL må ta hensyn til for å kunne nå sitt hovedmål om å bli ledende leverandør av isolasjon.

De interne analysene viser flere sterke sider ved ROCKWOOL. Ved å være den nest største aktøren i markedet, oppnår bedriften stordriftsfordeler som gir lavere enhetskostnader. Dette er en kritisk faktor for å kunne være konkurransedyktig i isolasjonsmarkedet. Bedriften sliter imidlertid med høye transportkostnader grunnet komprimeringsegenskapene til produktet. Produktegenskapene i seg selv gir konkurransefortrinn, men på grunn av for høyt CO2-utslipp i produksjonsprosessen klarer ikke bedriften å utnytte konkurransefortrinnet fullt ut. Kompetansen anses som meget god i form av erfaring og lav turnover. Bedriften bør imidlertid være oppmerksom på den høye gjennomsnittsalderen blant ansatte, og se på muligheter for å tilegne seg kunnskap utenfra. 
Et viktig område identifisert i den eksterne analysen er kundenes forhandlingsmakt og samfunnets fokus på et grønnere miljø. Gjennom økt bruk av BREEAM-sertifisering, stiller entreprenørene høye krav til produktene i form av lavt CO2-utslipp. Disse kravene klarer ikke ROCKWOOL å imøtekomme med nåværende produksjonsteknologi. Etter min vurdering bør bedriften investere i en ny smelteovn så fort som mulig for å kunne være konkurransedyktige i tiden fremover.

6. Strategiske handlingsalternativ
Basert på mine analyser har jeg kommet frem til fire strategiske handlingsalternativer som kan hjelpe ROCKWOOL med å nå sin visjon.

1. Status Quo
De interne analysene viser at en godt innarbeidet Lean-metode (Ropex) gjør ROCKWOOL til en veldreven og kostnadseffektiv bedrift. De har i flere år vært den nest største aktøren i et marked som domineres av noen få store isolasjonsprodusenter. Dette er et resultat av kontinuerlig fokus på å effektivisere verdikjeden og levering (fremstilling?) av de mest differensierte produktene i markedet. Etter min vurdering er det derfor naturlig å fortsette med samme strategi, men det bør nevnes at markedet er inne i en mer miljøvennlig utvikling som på sikt kan gi bedriften utfordringer med både å nå visjonen og beholde dagens markedsandel.
    
2. Go Green (investere i elektrisk smelteovn)
ROCKWOOL uttaler i sin nåværende visjon at de ønsker å bidra til et bedre miljø. Basert på de interne analysene, kommer det frem at de ikke gjør nok for å følge opp dette punktet i visjonen. I dagens produksjonsprosess benytter de seg av en kupolovn, som har et høyere CO2-utslipp enn eksempelvis Glava. De eksterne analysene identifiserer kundemakten som høy, delvis på grunn av økende bruk av BREEAM-sertifisering blant entreprenørene. Dette indikerer en bransjeutvikling hvor fokus på et grønnere miljø blir mer avgjørende for å kunne hevde seg. Skal ROCKWOOL på sikt kunne bli markedsledende, er det etter min vurdering nødvendig å vektlegge miljøet mer enn det som gjøres i dag. Bedriften bør investere i en ny el-smelteovn som kan redusere CO2-utslippet betraktelig. En omlegging fra kupolovn til fullskala el-ovn vil kreve en betydelig investering i form av finansiell kapital. Bedriften står i utgangspunktet godt rustet for å foreta en slik investering, men bør søke om støtte fra Enova for å få dekket deler av investeringsbeløpet. Dette bør også testes ut på én av fabrikkene, da investeringen medfører en viss risiko på bakgrunn av at teknologien bare har blitt testet ut i en mye mindre skala tidligere.  Det ligger også muligheter i transportmarkedet, som på sikt vil ta i bruk el-lastebiler. Det vil være viktig å implementere en miljøvennlig profil gjennom hele verdisystemet.  

3. Oppkjøp og fusjoner 
En rask måte for ROCKWOOL å utvide markedsandelen på er gjennom oppkjøp og fusjoner. De eksterne analysene viser til et modent marked som bærer preg av høy konkurranseintensitet. Det kan derfor bli krevende å nå målet om å bli ledende leverandør av isolasjon kun ved organisk vekst. En nærliggende konkurrent som vil være en potensiell kandidat til oppkjøp eller fusjon er Paroc. Begge bedriftene produserer isolasjon basert på steinull, noe som vil gjøre en eventuell integrasjon enklere, både når det gjelder koordinering av virksomhetens aktiviteter og ressurser, og ved å minimere usikkerhet blant de ansatte. Det er imidlertid viktig å påpeke at integrasjonsprosessen er krevende, og det bør derfor settes ned en egen prosjektgruppe som vil ha ansvar for å følge opp planene vedrørende integrasjonen. Fordeler ved oppkjøp eller en fusjon er forsterkning av stordriftsfordeler ved å kunne produsere i et større volum som gir lavere enhetskostnader (Teori fra s.240-242). Dette åpner også opp for en kostnadslederskapsstrategi hvor bedriften kan utfordre konkurrentene på pris.

4. Utvide geografisk
Som tidligere nevnt, operer ROCKWOOL i det norske og svenske markedet hvor fabrikkene og administrasjonen er lokalisert i Norge. Ved å utvide produksjonen og et salgsapparat til Sverige, kan det legge til rette for å kunne øke omsetningen i det svenske markedet. Nærhet til markedet er viktig for å skape relasjoner og bygge nettverk i det lokale markedet. Samtidig vil dette kunne redusere transportkostnadene, noe som er en betydelig kostnad i nåværende situasjon.
 
Det presiseres at alternativ 3 og 4 vil være strategier som må håndteres og godkjennes av konsernledelsen i ROCKWOOL international AS, og vil følgelig ikke bli diskutert videre.


7. Anbefalt strategi og implementering
På bakgrunn av analysene jeg har gjennomført, mener jeg at en kombinasjon av handlingsalternativene “Status Quo” og “Go Green” vil kunne hjelpe ROCKWOOL med å nå sin visjon. 

Økonomiske nøkkeltall viser til en bedrift som har prestert godt over tid, noe som er et resultat av flere år med fokus på å effektivisere verdikjeden. Implementeringen av Ropex har vært en viktig faktor i dette arbeidet, og det vil derfor være naturlig å fortsette med fokus på organisk vekst. På bakgrunn av markedsanalysen vil jeg imidlertid anbefale ROCKWOOL til å tillegge miljøet større vekt i deres nåværende strategi, hvor bedriften bør jobbe mot et mål om å bli en av markedets mest miljøvennlige aktører. Ved å investere i nytt produksjonsutstyr vil ROCKWOOL kunne utvide markedsandelen sin ved å bli en foretrukket leverandør blant entreprenørene, og på sikt utfordre Glava som markedsleder. 

For å evaluere strategien vil jeg benytte SAF-kriteriene; egnethet, aksept og gjennomførbarhet. (engelsk bok, teori s.236-241) Kriteriene sørger for å kvalitetssikre strategien. Den anbefalte løsningen samsvarer med markedsutviklingen og de største truslene bedriften står overfor i dag. ROCKWOOL sine interessenter vil med stor sannsynlighet være åpne for endringen. De ansatte vil måtte forholde seg til nye KPIer og måltall, men arbeidsoppgavene vil være de samme. Investeringen medfører risiko i form av en oppstartsperiode hvor tilpasning og utforskning blir viktig for å integrere den nye teknologien. Teknologien er ikke tatt i bruk i samme skala som ROCKWOOL tidligere. I tillegg er det en betydelig investering som vil være bedriftsøkonomisk ulønnsom de første årene i form av høyere avskrivninger og nevnte oppstartsfase, hvor det med stor sannsynlighet vil ta tid før optimal tilpasning i produksjonsprosessen nås. Prosjektet anses som gjennomførbart med støtte fra Enova. Hvis det ikke oppnås støtte fra Enova, presiseres det at det kreves finansiell støtte fra ROCKWOOL international AS.



