Basert på mine analyser har jeg kommet frem til fire strategiske handlingsalternativer som kan hjelpe ROCKWOOL med å nå sin visjon.

\subsection*{1. Fortsette med organisk vekst}
De interne analysene viser at en godt innarbeidet Lean-metode (Ropex) gjør ROCKWOOL til en veldreven og kostnadseffektiv bedrift. De har i flere år vært den nest største aktøren i et marked som domineres av noen få store isolasjonsprodusenter. Dette er et resultat av kontinuerlig fokus på å effektivisere verdikjeden og levering av de mest differensierte produktene i markedet. Etter min vurdering er det derfor en naturlig mulighet å fortsette med organisk vekst, men med større fokus på teknologiutvikling for å forbedre komprimeringsegenskapene til produktet og bevaring av kompetanse.

\subsection*{2. Go Green}
ROCKWOOL uttaler i sin nåværende visjon at de ønsker å bidra til et bedre miljø. Basert på de interne analysene, kommer det frem at de ikke gjør nok for å følge opp dette punktet i visjonen. I dagens produksjonsprosess benytter de seg av en kupolovn, som har et høyere CO2-utslipp enn flere av konkurrentene. De eksterne analysene identifiserer kundemakten som høy, delvis på grunn av økende bruk av BREEAM-sertifisering blant entreprenørene. Dette indikerer en bransjeutvikling hvor fokus på et grønnere miljø blir mer avgjørende for å kunne hevde seg. Skal ROCKWOOL på sikt kunne bli markedsledende, er det etter min vurdering nødvendig å vektlegge miljøet mer enn det som gjøres i dag. Bedriften bør investere i en ny el-smelteovn som kan redusere CO2-utslippet betraktelig. En omlegging fra kupolovn til fullskala el-ovn vil kreve en betydelig investering i form av finansiell kapital. Bedriften står i utgangspunktet godt rustet for å foreta en slik investering, men bør søke om støtte fra Enova for å få dekket deler av investeringsbeløpet. Dette bør også testes ut på én av fabrikkene, da investeringen medfører en viss risiko på bakgrunn av at teknologien bare har blitt testet ut i en mye mindre skala tidligere.  Det ligger også muligheter i transportmarkedet, som på sikt vil ta i bruk el-lastebiler. Det vil være viktig å implementere en miljøvennlig profil gjennom hele verdisystemet.  

\subsection*{3. Oppkjøp/Fusjon}
En rask måte for ROCKWOOL å utvide markedsandelen på er gjennom oppkjøp og fusjoner. De eksterne analysene viser til et modent marked som bærer preg av høy konkurranseintensitet. Det kan derfor bli krevende å nå målet om å bli ledende leverandør av isolasjon kun ved organisk vekst. En nærliggende konkurrent som vil være en potensiell kandidat til oppkjøp eller fusjon er Paroc. Begge bedriftene produserer isolasjon basert på steinull, noe som vil gjøre en eventuell integrasjon enklere, både når det gjelder koordinering av virksomhetens aktiviteter og ressurser, og ved å minimere usikkerhet blant de ansatte. Det er imidlertid viktig å påpeke at integrasjonsprosessen er krevende, og det bør derfor settes ned en egen prosjektgruppe som vil ha ansvar for å følge opp planene vedrørende integrasjonen. Fordeler ved oppkjøp eller en fusjon er forsterkning av stordriftsfordeler ved å kunne produsere i et større volum som gir lavere enhetskostnader \cite[s.~240-242]{FjeldstadogLunnan2018}. Dette åpner også opp for en kostnadslederskapsstrategi hvor bedriften kan utfordre konkurrentene på pris.

\subsection*{4. Geografisk utvidelse}
ROCKWOOL opererer i det norske og svenske markedet hvor fabrikkene og administrasjonen er lokalisert i Norge. Ved å utvide produksjonen og et salgsapparat til Sverige, kan det legge til rette for å kunne øke omsetningen i det svenske markedet. Nærhet til markedet er viktig for å skape relasjoner og bygge nettverk i det lokale markedet. Samtidig vil dette kunne redusere transportkostnadene, noe som er en betydelig kostnad i nåværende situasjon.

\indent \newline
Det presiseres at alternativ tre og fire vil være strategier som ROCKWOOL må bringe videre til konsernledelsen for håndtering og godkjenning.