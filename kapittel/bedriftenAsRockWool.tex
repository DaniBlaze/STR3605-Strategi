Jeg bestemte meg tidlig for å velge og skrive om AS ROCKWOOL, da jeg kjenner sjefen for fabrikken i Moss (Erik Ølstad). Dette ga meg muligheten til å holde en jevn dialog med en ansatt i bedriften som har meget god innsikt i hvordan bedriften drives og markedet den opererer i. Utover i prosessen fikk jeg anledning til å intervjue Lasse Scott (økonomiansvarlig) og Hans Joachim Motzfeldt (senior advisor). Ved å benytte en semistrukturert intervjuform, skapte dette en åpen samtale og anledning for intervjuobjektene til å reflektere rundt temaet og oppgaven. Økonomiske tall relatert til bedriften og dens konkurrenter er primært hentet fra proff.no. Annen informasjon er blitt hentet fra bedriftens hjemmeside og diverse artikler på nettet. Jeg har gjennom arbeidsprosessen aktivt benyttet analyser og metoder fra pensum.

\section{Om ROCKWOOL}
AS ROCKWOOL er et heleid norsk datterselskap av ROCKWOOL International A/S. Konsernet er verdens største steinullprodusent med ca 10.600 ansatte, salgskontor i over 35 land, og 28 steinullfabrikker og 11 andre steinullrelaterte fabrikker i 17 land. Datterselskapet består i dag av 240 ansatte fordelt på administrasjon i Oslo, salgsapparatet og to fabrikker lokalisert henholdsvis i Moss og Trondheim. ROCKWOOL baserer sin virksomhet på utvinning av vulkansk stein for å produsere produkter, systemer og løsninger innenfor byggisolasjon.

\indent \newline
Den første ROCKWOOL-fabrikken ble startet i 1937 i Danmark. Allerede året etter kom den første norske fabrikken i Larvik, og noen år senere utvidet de med fabrikk i Moss og Trondheim. I 1965 fusjonerte Elkem og ROCKWOOL, hvor Elkem i 1993 ble kjøpt ut av ROCKWOOL International A/S. ROCKWOOL har siden oppstarten produsert steinull, hvor selve produksjonsprosessen har vært relativt lik. Derimot ble det i 2002 investert en halv milliard kroner i produksjonsutstyr som ved hjelp av robotteknologi ga en helautomatisert produksjonsprosess. Dette har ført til mer enn en fordobling av produksjonen, og gitt helsegevinster ved at arbeidsoppgaver tilknyttet smelteprosessen har blitt erstattet av maskiner og datateknologi.

\indent \newline
AS ROCKWOOL har i lengre tid gjort det bra, og kan vise til sterke nøkkeltall i form av god likviditet og meget god lønnsomhet og soliditet. Bedriften har i dag en markedsandel på ca 26\% og omsatte for over 800 millioner kroner i 2017. Deres største konkurrent er Glava med en markedsandel på 40\% og en omsetning på over 1,3 milliarder kroner i samme år. Bransjen er preget av høye faste kostnader og store investeringer. Konkurransesituasjon er tøff, med et stadig økende fokus på miljøvennlige produkter og krav fra myndighetene. “I anbudsprosessen har AS ROCKWOOL uttalt at valget faller på leverandører som har lavest CO2-utslipp, da det gir en gevinst i ettertid for kundene.“

\section{Visjon}
Bedriftens visjon er å være ledende leverandør av isolasjon, der positivt bidrag til et bedre miljø og brannsikring skal være førende. Dette er basert på FN sine mål?

\section{Utfordringer}
Utfordringene ROCKWOOL står overfor er å finne nye måter å produsere på som er mer miljøvennlige for å kunne være konkurransedyktige også i fremtiden.

\section{Organisasjonskart}
\section{Forretningsidé}
\section{Strategiske mål}
