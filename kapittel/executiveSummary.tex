\textit{ROCKWOOL bør utvikle seg mot en grønn fremtid.}

\indent \newline
AS ROCKWOOL (heretter omtalt som ROCKWOOL) er et heleid norsk datterselskap av ROCKWOOL international A/S. ROCKWOOL er en isolasjonsprodusent som baserer sin virksomhet på utvinning av vulkansk stein. Visjonen deres er \textit{\textquotedblleft AS ROCKWOOL skal være ledende leverandør av isolasjon, der positivt bidrag til et bedre miljø og brannsikring skal være førende.\textquotedblright} Bedriften har i flere år levert gode resultater, og er markedets nest største aktør. Den siste tiden har de imidlertid opplevd en lavere prosentvis vekst, grunnet en utvikling i markedet hvor miljøet blir vektlagt mer enn tidligere.

\indent \newline
De interne analysene viser at ROCKWOOL har flere sterke og svake sider. Med en markedsandel på ca 26\%, oppnår bedriften stordriftsfordeler som gir lavere enhetskostnader. Gjennom Ropex fokuseres det kontinuerlig på effektivisering av verdikjeden. De har i utgangspunktet konkurransefortrinn i produktegenskapene, men klarer ikke å utnytte dette godt nok. Årsaken er nåværende produksjonsteknologi som gir et høyere CO2-utslipp enn de nærmeste konkurrentene. En annen svak side er høye transportkostnader grunnet komprimeringsegenskapene til produktet. Analysene viser videre at bedriften besitter høy kompetanse, i form av erfaring og lav turnover. Bedriften bør imidlertid være oppmerksom på en høy gjennomsnittsalder blant de ansatte. 

\indent \newline
Resultatene fra de eksterne analysene viser til en høy forhandlingsmakt hos kundene. En økende bruk av BREEAM-sertifisering blant entreprenørene fører til strengere produktkrav i form av lavt CO2-utslipp. Bransjeanalysen viser til et modent marked med lav vekst, og består av noen få store aktører. Konkurranseintensiteten anses som høy og vedvarende. 

\indent \newline
Basert på analysene har jeg identifisert fire strategiske handlingsalternativer som kan hjelpe ROCKWOOL med å nå sin visjon:

\begin{itemize}
\item\textbf{1. Fortsette med organisk vekst}
\indent \newline
Fokus på organisk vekst har så langt blitt vektlagt av ROCKWOOL. Flere år med solide økonomiske tall gjør det naturlig å fortsette med samme strategi. En godt innarbeidet Lean-metode (Ropex) og interne opplærings- og utdanningssystemer gjør ROCKWOOL til en veldreven og kostnadseffektiv bedrift. Fremtidige fokusområder bør være teknologiutvikling relatert til produktkomprimering og bevaring av kompetanse. 

\item[2.]\textbf{Go Green}
\indent \newline
Implementere en miljøvennlig profil gjennom hele verdisystemet. Investering i en fullskala el-ovn vil kunne redusere CO2-utslippet med ca. 80\% og gjøre ROCKWOOL til en foretrukken leverandør blant entreprenørene. På sikt bør bedriftens inngående- og utgående logistikk bestå av leverandører som benytter el-transport.

\item[3.]\textbf{Oppkjøp/Fusjon}
\indent \newline
Oppkjøp eller fusjon er en rask måte for ROCKWOOL å utvide markedsandelen sin på. Paroc anses som en potensiell kandidat med tanke på integrasjonsprosessen og koordinering av aktiviteter. Valget vil gi økte stordriftsfordeler og lavere enhetskostnader, og tilgang til andre operasjonelle synergier.

\item[4.]\textbf{Geografisk utvidelse}
\indent \newline
En fabrikkutvidelse i Sverige vil kunne øke omsetningen i det svenske markedet gjennom nærhet til markedet og reduserte transportkostnader.
\end{itemize}
\indent
Det presiseres at alternativ tre og fire vil være strategier som ROCKWOOL må bringe videre til konsernledelsen for håndtering og godkjenning.

\indent \newline
Basert på analysene mener jeg en kombinasjon av handlingsalternativene “Fortsette med organisk vekst” og “Go Green” vil kunne hjelpe ROCKWOOL med å nå sin visjon. Den anbefalte løsningen samsvarer med markedsutviklingen og de største truslene bedriften står overfor i dag. Ved å kombinere nåværende strategi med en grønn utvikling vil de kunne bli en foretrukket leverandør blant entreprenørene, og dermed utvide markedsandelen på sikt. Investering i en ny el-smelteovn vil være bedriftsøkonomisk ulønnsom de første årene, grunnet en prøveperiode med implementering av teknologien. Investeringen vil kreve betydelige finansielle midler og det anbefales derfor å søke om støtte fra Enova. Ved avslag må finansiell støtte hentes fra ROCKWOOL international A/S. 

\indent \newline
Link til video: 
